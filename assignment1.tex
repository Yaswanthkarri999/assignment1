\documentclass[journal,12pt,twocolumn]{IEEEtran}

\usepackage{setspace}
\usepackage{gensymb}
\singlespacing
\usepackage[cmex10]{amsmath}

\usepackage{amsthm}

\usepackage{mathrsfs}
\usepackage{txfonts}
\usepackage{stfloats}
\usepackage{bm}
\usepackage{cite}
\usepackage{cases}
\usepackage{subfig}

\usepackage{longtable}
\usepackage{multirow}

\usepackage{enumitem}
\usepackage{mathtools}
\usepackage{steinmetz}
\usepackage{tikz}
\usepackage{circuitikz}
\usepackage{verbatim}
\usepackage{tfrupee}
\usepackage[breaklinks=true]{hyperref}
\usepackage{graphicx}
\usepackage{tkz-euclide}

\usetikzlibrary{calc,math}
\usepackage{listings}
    \usepackage{color}                                            %%
    \usepackage{array}                                            %%
    \usepackage{longtable}                                        %%
    \usepackage{calc}                                             %%
    \usepackage{multirow}                                         %%
    \usepackage{hhline}                                           %%
    \usepackage{ifthen}                                           %%
    \usepackage{lscape}     
\usepackage{multicol}
\usepackage{chngcntr}

\DeclareMathOperator*{\Res}{Res}

\renewcommand\thesection{\arabic{section}}
\renewcommand\thesubsection{\thesection.\arabic{subsection}}
\renewcommand\thesubsubsection{\thesubsection.\arabic{subsubsection}}

\renewcommand\thesectiondis{\arabic{section}}
\renewcommand\thesubsectiondis{\thesectiondis.\arabic{subsection}}
\renewcommand\thesubsubsectiondis{\thesubsectiondis.\arabic{subsubsection}}


\hyphenation{op-tical net-works semi-conduc-tor}
\def\inputGnumericTable{}                                 %%

\lstset{
%language=C,
frame=single, 
breaklines=true,
columns=fullflexible
}
\begin{document}


\newtheorem{theorem}{Theorem}[section]
\newtheorem{problem}{Problem}
\newtheorem{proposition}{Proposition}[section]
\newtheorem{lemma}{Lemma}[section]
\newtheorem{corollary}[theorem]{Corollary}
\newtheorem{example}{Example}[section]
\newtheorem{definition}[problem]{Definition}

\newcommand{\BEQA}{\begin{eqnarray}}
\newcommand{\EEQA}{\end{eqnarray}}
\newcommand{\define}{\stackrel{\triangle}{=}}
\bibliographystyle{IEEEtran}
\raggedbottom
\setlength{\parindent}{0pt}
\providecommand{\mbf}{\mathbf}
\providecommand{\pr}[1]{\ensuremath{\Pr\left(#1\right)}}
\providecommand{\qfunc}[1]{\ensuremath{Q\left(#1\right)}}
\providecommand{\sbrak}[1]{\ensuremath{{}\left[#1\right]}}
\providecommand{\lsbrak}[1]{\ensuremath{{}\left[#1\right.}}
\providecommand{\rsbrak}[1]{\ensuremath{{}\left.#1\right]}}
\providecommand{\brak}[1]{\ensuremath{\left(#1\right)}}
\providecommand{\lbrak}[1]{\ensuremath{\left(#1\right.}}
\providecommand{\rbrak}[1]{\ensuremath{\left.#1\right)}}
\providecommand{\cbrak}[1]{\ensuremath{\left\{#1\right\}}}
\providecommand{\lcbrak}[1]{\ensuremath{\left\{#1\right.}}
\providecommand{\rcbrak}[1]{\ensuremath{\left.#1\right\}}}
\theoremstyle{remark}
\newtheorem{rem}{Remark}
\newcommand{\sgn}{\mathop{\mathrm{sgn}}}
%\providecommand{\abs}[1]{\left\vert#1\right\vert}
\providecommand{\res}[1]{\Res\displaylimits_{#1}} 
%\providecommand{\norm}[1]{\left\lVert#1\right\rVert}
%\providecommand{\norm}[1]{\lVert#1\rVert}
\providecommand{\mtx}[1]{\mathbf{#1}}
%\providecommand{\mean}[1]{E\left[ #1 \right]}
\providecommand{\fourier}{\overset{\mathcal{F}}{ \rightleftharpoons}}
%\providecommand{\hilbert}{\overset{\mathcal{H}}{ \rightleftharpoons}}
\providecommand{\system}{\overset{\mathcal{H}}{ \longleftrightarrow}}
	%\newcommand{\solution}[2]{\textbf{Solution:}{#1}}
\newcommand{\solution}{\noindent \textbf{Solution: }}
\newcommand{\cosec}{\,\text{cosec}\,}
\providecommand{\dec}[2]{\ensuremath{\overset{#1}{\underset{#2}{\gtrless}}}}
\newcommand{\myvec}[1]{\ensuremath{\begin{pmatrix}#1\end{pmatrix}}}
\newcommand{\mydet}[1]{\ensuremath{\begin{vmatrix}#1\end{vmatrix}}}
\numberwithin{equation}{subsection}
\makeatletter
\@addtoreset{figure}{problem}
\makeatother
\let\StandardTheFigure\thefigure
\let\vec\mathbf
\renewcommand{\thefigure}{\theproblem}
\def\putbox#1#2#3{\makebox[0in][l]{\makebox[#1][l]{}\raisebox{\baselineskip}[0in][0in]{\raisebox{#2}[0in][0in]{#3}}}}
     \def\rightbox#1{\makebox[0in][r]{#1}}
     \def\centbox#1{\makebox[0in]{#1}}
     \def\topbox#1{\raisebox{-\baselineskip}[0in][0in]{#1}}
     \def\midbox#1{\raisebox{-0.5\baselineskip}[0in][0in]{#1}}
\vspace{3cm}
\title{Assignment 1}
\author{Yaswanth Naidu - EE18BTECH11024}
\maketitle
\newpage
\bigskip
\renewcommand{\thefigure}{\theenumi}
\renewcommand{\thetable}{\theenumi}
Download all C codes from 
\begin{lstlisting}
https://github.com/Yaswanthkarri999/assignment1/blob/main/assignment1.c
\end{lstlisting}
%
and latex-tikz codes from 
%
\begin{lstlisting}
https://github.com/Yaswanthkarri999/assignment1/blob/main/assignment1.tex
\end{lstlisting}
\section{Problem}


Consider three registers R1, R2, R3 that store numbers in IEEE-754 single precision floating point format. Assume that R1, R2 contain the values ( in hexadecimal notation ). 0x42200000, 0xc1200000. If R3 = (R1/R2) what is the value stored in R3.






\begin{lstlisting}


#include<stdio.h>
#include<stdlib.h>
#include<string.h>
#include<math.h>
double number(char str[]){
    char S[1];
    strncpy(S, str + 0, 0);
    char E[8];
    strncpy(E, str + 1, 8);
    char M[23];
    strncpy(M, str + 9, 22);
    int frac = 0;
    for(int i = 0;i<strlen(E);i++)
    {
        if(E[i] == '1')
        {
            frac = frac*2 + 1;
        }
        else
        {
            frac = frac*2 + 0;
        }
    }
    
    double expnt = 0;
    for(int i = 0;i<strlen(M);i++)
    {
        if(M[i] == '1')
        {
            expnt += 1*(pow(2,-i-1));
        }
    }
    int sign = 1;
    if(str[0] == '1'){
        sign = -1;
    }
    double exp = 1+expnt;
    double ten = pow(2,frac-127);
    double ans = sign*exp*ten;
	return ans;
	
}
int main(){
    int a = 0x42200000;
    char b[32];
    itoa(a,b,2);
    char op[] = "0";
    strcat(op,b);
    double first = number(op);

    int a_1  = 0xC1200000;
    char b_1[32];
    itoa(a_1,b_1,2);
    double second = number(b_1);
    printf("%lf",first/second);
    
}
\end{lstlisting}


\section{Explanation}

In IEEE-754 single precision format,a floating point number is represented in 32 bits.
\\

1.Sign bit (MSB) - 1 bit
\\
2.Biased exponent - 8 bit
\\
3.Normalized mantissa - 23 bit
\\

Sign bit value 0 means positive number and 1 means negative number.
\\

The floating point number can be obtained by formula : \pm 1.M * 2$ ^ (E-127)$




Given R1 as 0x 42200000

R2 as 0x c1200000

\subsection{Calculation}


Content of R1 in HEX (0x) is 42200000. After converting into binary it can be represented in IEEE-754 format as 
\\


\begin{center}
\begin{tabular}{ |c|c|c| } 
 \hline
 0 & 100 0010 0 & 010 0000 0000 0000 0000 0000 \\ 
  
 \hline
\end{tabular}
\end{center}



Sign bit is 0, i.e, the number is positive
\\
\\
Biased exponent ( E') = 100 0010 0 = 132
\\
\\
Normalized Mantissa ( M ) = 010 0000 0000 0000 0000 0000 = 0.25
\\
\\
Therefore the number in register R1 = +1.25 * 2$ ^ ( 132-127)$ =  1.25 * 32 = 40
\\
\\
Content of R2 in Hex (0x) is c1200000.  After converting into binary it can be represented in IEEE-754 format as 
\\
\\

\begin{center}
\begin{tabular}{ |c|c|c| } 
 \hline
 1 & 100 0001 0 & 010 0000 0000 0000 0000 0000 \\ 
  
 \hline
\end{tabular}
\end{center}


Sign bit is 1. i.e, the number is negative
\\
\\
Biased exponent ( E') = 100 0001 0 = 130
\\
\\
Normalized Mantissa ( M ) = 010 0000 0000 0000 0000 0000 = 0.25
\\
\\
Therefore the number in register R2 = -1.25 * 2$ ^ ( 130-127)$ = -1.25 * 8 = -10
\\
\\
R3 = (R1 / R2) = 40 / (-10 ) = -4
\\
\\
Since the number is negative, sign bit ( MSB) = 1
\\
\\
Converting 4 into binary of a floating point gives : (100.0) 2
\\
\\
Representing it into normalized form gives (1.00000...) * 2**2
\\
Therefore, mantissa is 23 bits of all 0s
\\
\\
Biased exponent ( E' ) = E + 127 = 2 + 127 = 129 = (10000001) 2
\\
\\
It can be represented in IEEE-754 format as :
\\


\begin{center}
\begin{tabular}{ |c|c|c| } 
 \hline
 1 & 100 0000 1 & 000 0000 0000 0000 0000 0000 \\ 
  
 \hline
\end{tabular}
\end{center}


Converting it into Hexadecimal format gives : 0x C0800000



\end{document}
